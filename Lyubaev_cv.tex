\documentclass{resume}

\begin{document}

\fontfamily{ppl}\selectfont

\noindent
\begin{tabularx}{\linewidth}{@{}m{0.8\textwidth} m{0.2\textwidth}@{}}
{
    {\Large{Daniil Lyubaev}} \newline
    {\small{Saint-Petersburg, Russia}} \newline
    \small{
        \clink{ \footnotesize
            Email: lyubaevda@gmail.com \newline 
            \href{https://github.com/eqimd}{GitHub: github.com/eqimd}
        } \newline
    }
} & 
{
    %\hfill
    %\includegraphics[width=2.8cm]{images/gr.png}
}
\end{tabularx}
\begin{tabularx}{\linewidth}{@{}m{1\textwidth} m{0.2\textwidth}@{}}
\csection{\bf Education}{\small
    \begin{itemize}
        \item {\bf B.S. Mathematics, Algorithms and Data Analysis}

        Saint-Petersburg State University,
        Mathematics and Computer Sciences department 
        
        {\footnotesize 3rd year, GPA: 4.3 / 5}
        \item
        Attended courses:
        \begin{itemize}
            \item Kotlin by Computer Science Center.
            \item Python by Nikita Sobolev.
        \end{itemize}
    \end{itemize}
}
\end{tabularx}
\begin{tabularx}{\linewidth}{@{}m{1\textwidth} m{0.2\textwidth}@{}}
\csection{\bf Programming experience}{\small
    \begin{itemize}
        \item MCS Android App (Android, Kotlin) \newline
            {\footnotesize Created mobile app with login, register, posts, userlist,
            etc., with mock API and real API}
        \item ldd (C++) \newline
            {\footnotesize Created a replica of ldd app that reads dynamic ELF section and shows dependent libraries}
        \item Django learning process (Python, Django) \newline
            {\footnotesize Created a Django app with courses, students, lectures, calls to database without $N+1$ -- problem}
        \item Socket HTTP file-transer server (Python, sockets) \newline
            {\footnotesize Made a socket server for transfering files with HTTP header}
        \item Huffman archiver (C++) \newline
            {\footnotesize Implemented Huffman compression/uncompression algorithm}
        \item Contribution to open-source NeoVim plugin {AirLatex} (Python) \newline
            {\footnotesize Created pull requests on bug fixes in Python code}
        \item Chat application (Python) \newline
            {\footnotesize Taught myself Twisted and PyQt Python libraries and created primitive echo-server/client app}
    \end{itemize}
}
\end{tabularx}
\begin{tabularx}{\linewidth}{@{}m{1\textwidth} m{0.2\textwidth}@{}}
\csection{\bf Knowledge and skills}{\small
    \begin{itemize}
        \item Programming: Kotlin, Android, Python, C/C++.
        \item Have basic knowledge of operating systems, had experience with Django, PyGame, SFML.
        \item Languages: Russian -- Native; English -- Intermediate.
    \end{itemize}
}
\end{tabularx}
% \begin{tabularx}{\linewidth}{@{}m{1\textwidth} m{0.2\textwidth}@{}}
% \csection{\bf Other experience/knowledge/interests}{\small
%     \begin{itemize}
%         \item Proved upper and lower bounds in the article ``{\it Transforming reversible two-way finite automata to one-way deterministic}''.
%             A joint paper with \href{https://users.math-cs.spbu.ru/~okhotin/}{Alexander Okhotin} is on the way.
%         \item Interested in formal languages theory (automata, grammars), TCS.
%     \end{itemize}
% }
% \end{tabularx}
\end{document}

